% mnras_template.tex
%
% LaTeX template for creating an MNRAS paper
%
% v3.0 released 14 May 2015
% (version numbers match those of mnras.cls)
%
% Copyright (C) Royal Astronomical Society 2015
% Authors:
% Absem Jibrail (The University of Sydney)

% Change log
%
% v3.0 May 2015
%    Renamed to match the new package name
%    Version number matches mnras.cls
%    A few minor tweaks to wording
% v1.0 September 2013
%    Beta testing only - never publicly released
%    First version: a simple (ish) template for creating an MNRAS paper

%%%%%%%%%%%%%%%%%%%%%%%%%%%%%%%%%%%%%%%%%%%%%%%%%%
% Basic setup. Most papers should leave these options alone.
\documentclass[a4paper,fleqn,usenatbib]{mnras}

% MNRAS is set in Times font. If you don't have this installed (most LaTeX
% installations will be fine) or prefer the old Computer Modern fonts, comment
% out the following line
\usepackage{newtxtext,newtxmath}
% Depending on your LaTeX fonts installation, you might get better results with one of these:
%\usepackage{mathptmx}
%\usepackage{txfonts}

% Use vector fonts, so it zooms properly in on-screen viewing software
% Don't change these lines unless you know what you are doing
\usepackage[T1]{fontenc}
\usepackage{ae,aecompl}

%%%%% AUTHORS - PLACE YOUR OWN PACKAGES HERE %%%%%

% Only include extra packages if you really need them. Common packages are:
\usepackage{graphicx}	% Including figure files
\usepackage{amsmath}	% Advanced maths commands
\usepackage{nccmath}
\usepackage{amssymb}	% Extra maths symbols
\usepackage{verbatim} % To allow for commenting out of blocks of text using comment
\usepackage{subcaption}
\captionsetup{compatibility=false}

%%%%%%%%%%%%%%%%%%%%%%%%%%%%%%%%%%%%%%%%%%%%%%%%%%

%%%%% AUTHORS - PLACE YOUR OWN COMMANDS HERE %%%%%

% user defined commands
% units
\def \Msun{\ {\rm M_\odot}}
\def \Msunh{\ h^{-1}{\rm M_\odot}}
\def \Mpc{{\rm Mpc}}
\def \kpc{{\rm kpc}}
\def \Mpch{\ h^{-1}{\rm Mpc}}
\def \kpch{\ h^{-1}{\rm kpc}}
\def \Gpch{\ h^{-1}{\rm Gpc}}
\def \lcdm{$\Lambda$CDM }
\def \lwdm{$\Lambda$WDM }
\def \qcdm{$\phi$CDM }
\newcommand{\dedm}[1]{$\phi(\beta_o=#1)$CDM}

% reference commands
\newcommand{\Eqref}[1]{Eq.~(\ref{#1})}
\newcommand{\Figref}[1]{Fig.~\ref{#1}}
\newcommand{\Secref}[1]{\S\ref{#1}}  
\newcommand{\Tableref}[1]{Table~\ref{#1}}


% for emphasizing comments
%\newcommand{\PJE}[1]{{\bf\color{Red}PJE-{#1}}}

% number of cosmologies
\newcommand{\ncosmo}{3}

% Please keep new commands to a minimum, and use \newcommand not \def to avoid
% overwriting existing commands. Example:
%\newcommand{\pcm}{\,cm$^{-2}$}	% per cm-squared


%%%%%%%%%%%%%%%%%%%%%%%%%%%%%%%%%%%%%%%%%%%%%%%%%%

%%%%%%%%%%%%%%%%%%% TITLE PAGE %%%%%%%%%%%%%%%%%%%

% Title of the paper, and the short title which is used in the headers.
% Keep the title short and informative.
%\title[\textit{N}-Body Studies of Non-Standard Cosmologies]{ \textit{N}-Body Studies of Non-Standard Cosmologies:\\ Spin Correlations and Large-Scale Structures }
\title[Seeking cosmological signatures in simulations]{ Seeking Cosmological Signatures in Simulations:\\ Spin Correlations and Large-Scale Structures }
% The list of authors, and the short list which is used in the headers.
% If you need two or more lines of authors, add an extra line using \newauthor
\author[A. W. Jibrail et al.]{
Absem W. Jibrail,$^{1}$\thanks{E-mail: ajib0457@uni.sydney.edu.au (AWJ)}
Geraint F. Lewis$^{1}$
and Pascal J. Elahi$^{2}$
\\
% List of institutions
$^{1}$Sydney Institute for Astronomy, School of Physics, A28, The University of Sydney, NSW, 2006, Australia\\
$^{2}$International Centre for Radio Astronomy Research (ICRAR), The University of Western Australia, 35 Stirling Hwy, \\
Crawley, Western Australia 6009, Australia}

% These dates will be filled out by the publisher
\date{Accepted XXX. Received YYY; in original form ZZZ}

% Enter the current year, for the copyright statements etc.
\pubyear{2018}

% Don't change these lines
\begin{document}
\label{firstpage}
\pagerange{\pageref{firstpage}--\pageref{lastpage}}
\maketitle

% Abstract of the paper
\begin{abstract}
Despite the Standard Cosmological Model's (\lcdm) predictive accuracy on the largest scales (beyond galactic scales) of the universe, there are conceptual and observational shortfalls on galactic scales which loom large within dark sector cosmology. Problems of \textit{fine-tuning} the vacuum energy density $\Lambda$, a Dark Energy (DE)- Dark Matter (DM) \textit{coincidence} problem and the Missing Satellite problem, have inspired theorists to posit and study - the latter using $N$-body simulations - non-standard cosmologies in an effort to alleviate such shortfalls. This is a comparative study between a Warm Dark Matter (\lwdm), a coupled DM-DE cosmology (CDE), an uncoupled Quintessence (\qcdm) along with the fiducial \lcdm, conducted with the evolution of DM Halos Angular Momentum. Comparing the spin-filament alignment correlations as functions of mass at various smoothing scales and from z=0 to z=2.98, we find no systematic differences between cosmologies. 

%This is a simple template for authors to write new MNRAS papers. The abstract should briefly describe the aims, methods, and main results of the paper. It should be a single paragraph not more than 250 words (200 words for Letters). No references should appear in the abstract.
\end{abstract}

% Select between one and six entries from the list of approved keywords.
% Don't make up new ones.
\begin{keywords}
cosmology: simulations -- cosmology: large-scale structure of the universe -- dark matter -- dark energy
\end{keywords}

%%%%%%%%%%%%%%%%%%%%%%%%%%%%%%%%%%%%%%%%%%%%%%%%%%

%%%%%%%%%%%%%%%%% BODY OF PAPER %%%%%%%%%%%%%%%%%%

\section{Introduction}\label{intro}

The Cosmic Microwave Background (CMB) reveals an anisotropic temperature distribution within the early universe which is thought to have occurred via quantum fluctuations during the period of Inflation \citep{Guth_82}. Causing minor inhomogeneities, these fluctuations cascaded, post-inflation through the process of gravitational instability, into what we see today as knots, filaments, sheets and voids. We now understand the Large-Scale Structure (LSS) are connected like a Cosmic Web, the structure being hierarchical and molded by the dark sector in its formation and evolution. The most successful model to date of the universe is \lcdm. This has been reaffirmed time and again; large-scale observations such as the anisotropies in the Cosmic Microwave Background (CMB) \citep{Bennett_13,Plank_14b,Plank_16}, features in the LSS \citep{Abazajian_09}, Baryonic Acoustic Oscillations (BAO) \citep{Beutler_11} and weak lensing \citep{Kilbinger_13}. Furthermore, constraints from observations have supported the CDM scenario \citep{Bertone_05,Petraki_13} and have well-motivated candidates from particle physics over more energetic dark matter such as Hot Dark Matter. Also DE, in the form of a cosmological constant, seems to work well to predict the late time accelerated expansion of the universe \citep{Suzuki_12}
\\
Despite all of its success, \lcdm suffers from various observational inaccuracies as well as conceptual shortfalls regarding the dark sector physics, serving as the motivation of this research. Underlying issues within \lcdm have motivated non-standard cosmological models including \lwdm, which poses to alleviate the Missing-Satellite problem, where \lcdm produces too many satellites around central galaxies \citep{Klypin_99,Moore_99}. Although some have proposed that this is not an issue of \lcdm but rather it is to do with the limitations of Dark matter simulations \citep{Wetzel_16} or feedback processes \citep[e.g][]{Bullock_00}. \qcdm poses to alleviate the \textit{fine-tuning} problem of the initial value for vacuum energy density by substituting the cosmological constant by a scalar field \citep{Tsujikawa_13}. There are also conceptual problems with \lcdm, one donned the coincidence problem states that the energy-density of DM and DE are similar at present day, which is unlikely given the age of the universe, thus perhaps there may be some inter-dependence within the dark sector that alleviates the otherwise unlikely coincidence. This gives rise to a coupled dark sector cosmology (CDE), see \citet{Bull_16} for a review on \lcdm shortfalls. 
\\
Tidal Torque Theory (TTT) describes the procurement of the initial Angular Momentum of dark matter halos, that is through the influence of the tidal field, which has a great influence on the progenitor before turn-around \citep[Holye 1949][]{Peebles_69,Zeldovich_70} 
It is argued that since the tidal field is the manifestation of the cosmological environment, and this field predominantly, although arguably, is responsible for the acquisition of initial Halo spins (spin refers to the halo unit vector direction of Angular Momentum hereafter). It is argued that the signatures of a cosmology are imprinted on the spins of galaxies \citep{Lee_pen_00} and thus spin may be a suitable probe of cosmology. In contrast, it has been conjectured that since over-dense regions such as sheets, filaments and clusters are susceptible to non-linear physics, these cosmological signatures are possibly erased or otherwise masked, placing doubt on whether halo spin is a good probe for cosmology.
\\
Nonetheless there are numerous studies which investigate the  LSS of \lcdm by using spin alignment and evolution as a probe within simulations. Studies such as \citet{Faltenbacher_02,Calvo_07,Zhang_09} and \citet{Hahn_07} find that there are clear correlation signals between LSS axes and halo spins, which can be linked to the formation of LSS and its influence on the spin and shapes of halos. Specifically, \citet{Zhang_09} find that the spin and shape of halos within filaments of M $\leq$ $10^{13}$ $h^{-1}$ \(M_\odot\) are aligned with the filament axes. Furthermore they find that the alignments are strengthened for halos which are closer to more massive node (cluster) halos. They interpret this as 'transit large-scale environment impact' that is a transit from two-dimensional filaments to three-dimensional clusters.
\citet{Trowland_13} conducts a spin evolution study using the Millennium simulations and finds that at low redshift, low mass DM halo spins ( $\sim$$10^{11.6}$ to $10^{12.9}$ $h^{-1}$ \(M_\odot\)) tend to be aligned to their filaments whereas high mass halos ( $\sim$ $10^{12.9}$ to $10^{13.4}$ $h^{-1}$ \(M_\odot\)) tend to be orthogonal in their alignment, suggesting that TTT may not be the sole mechanism for spin acquisition.  

The alignment of halo spin with LSS is not exclusive to filamentary structure, but is also seen within sheets and clusters, where there is a general alignment between halo spin and the e2 (initial intermediate tidal tensor) axes, also known as the axes of slowest collapse. \citep{Libeskind_12,Dubois_14,Calvo_14,Kang_15,Wang_17,Veena_18} 

\citet{Bond_96,Codis_12,Pichon_16} tell the story of how the spin evolution of halos are part in parcel of the pancaking effect whereby the fastest collapsing axis e3 causes mergers and accretion along this axis and explaining the alignments found. The horizon simulations \citep{Dubois_14,Welker_14} also show evidence of bulk flow as does \citet{Trowland_13} which explains the spin flip of high mass halos by undergoing major mergers, whereas low mass halos being less likely to undergo mergers retain their parallel spin to the slowest collapsing axis.

These alignments have not only been found in simulation but also in through observation \citep{Jones_10}. \citet{Pen_00} also shows tentative alignment in spirals. 
\citet{Lee_Erdogdu_07} show that galaxies in the Tully-Fisher catalog are weakly orthogonal with their environment, stating an average correlation coefficient (c) value of $\bar{c}=0.084\pm 0.014$ (where c=0 represents a random alignment) to their LSS with 99.9$\%$ confidence (based on Kolmogorov-Smirnov statistic) that the null hypothesis of no spin-shear correlation is rejected.
\\
There is much simulation-based research in cosmological model comparison. From studies such as \citet{Elahi_14} finding that above the suppression scale, on the scales of galaxy clusters, there are marked differences between \lcdm and \lwdm. \citet{Elahi_15} shows little to no difference between \lcdm and the coupled cosmologies, judging by the lack of systematic differences between spin parameter and satellite alignment distributions across cosmologies. \citet{Adermann_18} finds that there is a potentially observable difference between the volume distribution of voids between \lcdm and CDE at low redshift. Also, differences are found between the densities of voids for each cosmology, being emptiest in \qcdm and densest within \lcdm. \citet{Carlesi_14a} finds, using hydrodynamical simulations, that a self interacting quintessence model (uDE) provides a higher concentration of halos within the LSS as compared to their fiducial \lcdm and other cosmologies compared. \citet{Carlesi_14b} find a weak coupling between the spin, triaxiality and virialisation and the cosmology dark sector types. \citet{Smith_11} find that WDM model suppresses the halo mass function by 50$\%$ for masses 100 times the free-streaming mass scale. \citet{Watts_17} finds there are higher cluster abundances and lower void abundances within the \qcdm with respect to \lcdm. 

Aside from simulations and astronomical observations, there are active fields of research which hunt for dark matter particles, such as Weakly-Interacting Massive Particles (WIMPS)\citep{Cerde_10}, others explain away dark matter in the form of Massive Compact halo Objects (MACHOS) \citep{Alcock_00} or Modified Newtonian Dynamics (MOND) \citep{Milgrom_15}. There are other  Dark Matter and Dark energy alternatives, see \citet{Mannheim_06} for a review. 
\\
Section \ref{cosmos} of this paper will introduce the main components of the Non-Standard Cosmologies considered within this study. Section \ref{method} is the methodology of this investigation which details the simulations, structure classification and halo alignment. This is followed by Section \ref{results} presenting and elucidating the main findings of the investigation.Section \ref{discussion} contains the discussion of the results and Section \ref{conclusion} summarizes and concludes the findings. 
\begin{comment}
Observations of Supernovae type Ia are have been used as a standard candle to trace the nature and evolution of the universe. The FRW model of cosmology (derived from General Relativity for a perfect fluid, whereby the universe is assumed to be isotropic and homogeneous) is fit to these observations and it is found that \lcdm is an accurate cosmological model in that 1. the energy density resembles the critical density $\rho_{0}\approx 10^{-26}kg/m^{3}$ concluding a flat universe. 2. the matter energy density in the universe $\Omega_{M}^{flat}=0.28^{+0.09}_{-0.08}$ including dark matter 3. that the present accelerated expansion is owed to a cosmological constant $\Omega_{\Lambda}=0.7$, which recently (since z$\approx$0.37) became the dominant energy density of the universe. \citep{Riess_98,Perlmutter_99}
\end{comment}
\begin{comment}
According to linear theory and the Zel'dovich approximation \citep{Zeldovich_70} evolution of the cosmic web seemed to have begun from over-dense regions which collapsed along their shortest axes into sheets, then from sheets to filaments and finally into knots, totaling to make the cosmic web. But this large-scale collapse into these LSS provides the tidal field (also known as the bulk flow) unto which DM halos form and gain their initial spin . 
\end{comment}

\section{Non-Standard Cosmological Models}\label{nonstdcosmos}
The following section will outline the alternative models considered in this study and their altered dark sector physics. DM and DE are altered in their nature and interactions in an effort to alleviate such small-scale discrepancies found between observation and \lcdm but also conceptual problems of fine-tuning and DM-DE coincidence. 

\subsection{Uncoupled Quintessence (\qcdm)}\label{quint}
The general form of the Langrangian which describes the scalar field of \qcdm model
\begin{ceqn}
\begin{equation}
L=\int d^{4}x\sqrt[]{-g}\big(-\frac{1}{2}\partial_{\mu}\partial^{\mu}\phi+V(\phi)+m(\phi)\psi_{m}\bar{\psi}_{m}\big)
\end{equation}
\end{ceqn}
includes a kinetic term, potential term $V(\phi)$ and an interaction term $(\psi_{m})$ with Dark Matter.

The \qcdm model includes no direct interaction between Dark Matter and the quintessence field, thus $m(\phi)=m_{0}$. But Dark Energy - usually represented by a cosmological constant $\lambda$ - is replaced with a time-dependent, evolving scalar field $\phi$(in units of Plank mass) whereby regions of the universe have the opportunity to expand independently, thus alleviates the cosmological constant problem \citep{Joyce_15}.
\begin{ceqn}
\begin{equation}
V(\phi)=V_{0}\phi^{-\alpha}\label{rp}
\end{equation}
\end{ceqn}
As for the potential term $V(\phi)$, this cosmology uses the Ratra-Peebles Potential  \citep{Ratra_88},where $V_{0}$ and $\alpha$ are observationally fitted constants. This potential term is a contrived field potential, so, at late times the quintessence field dominates the energy budget of the universe, and at early times it 'tracks' the energy density of matter and radiation akin to observations \citep{Joyce_15}.

\subsection{Coupled Dark Energy (CDE)}\label{cde}
We also investigate a Coupled dark sector cosmology, whereby the DM decay resolves the coincidence problem of \lcdm. The interaction is executed via a non-zero interaction term $m(\phi)=m_{0}e^{-\beta(\phi)\phi}$ We consider two coupled cosmologies which can be distinguished by their coupling terms of $\beta(\phi)=\beta_{0}=0.05$ and $\beta(\phi)=\beta_{0}=0.099$. $\beta(\phi)$ are chosen to test the boundaries of allowed coupling with an effort to maximize any observational differences between the fiducial \citep{Pettorino_12}. Coupling allows for DM particles to decay into the DE scalar field $\phi$ resulting in an additional frictional force felt by the DM particles. This extra force ultimately effects the evolution of the density perturbation amplitude for DM, which in turn alters the baryon fraction within cluster-sized halos \citep{Baldi_10} 

\subsection{Warm Dark Matter  ($\Lambda$WDM)}\label{wdm}
The \lcdm is extremely successful on large scales, thus WDM aims to merely modify the \lcdm model in a effort to dampen small-scale structure of the universe. WDM features Dark Matter particles which move at relativistic velocities, increasing the length of the free-streaming of particles, which will smooth out over-densities and suppress structure formation at scales smaller than the co-moving scale\citep{Bode_01}.
The free-streaming limit of the WDM particle has been confined via observations of the Lyman-alpha forest, to a lower limit mass $m_{WDM}\gtrapprox$ 3.3KeV \citep{Viel_13} Despite this lower limit, the mass we assign to our WDM model has particle energy = 2keV to exaggerate the cosmological effects. By multiplying the initial power spectrum by a transfer function, we truncate structure formation at the scale of 0.15 $h^{-1}$ Mpc \citep{Bode_01}

\section{Methodology}\label{SiriusBlack}\label{method}

\begin{comment}
\begin{table}
	\centering
	\caption{Cosmological parameters for reference cosmology ($\Lambda$CDM) as well as WMD \& \qcdm}
	\label{tab:1}
	\begin{tabular}{lccr} % four columns, alignment for each
		\hline
		Model & $\Omega_{m}$ & $\Omega_{b}$ & $\sigma_{g}$  \\
		\hline
		$\Lambda$CDM(ref.) & 0.3175 & 0.049 & 0.83\\
		WDM & $512^{3}$ & 8.16$\times$$10^{10}$ & 8 \\
		\qcdm & $512^{3}$ & 8.16$\times$$10^{10}$ & 9 \\
		\hline
	\end{tabular}
\end{table}
\end{comment}

\subsection{\textit{N}-Body Simulations}\label{nbodysim}
Prior to detailing the N-body simulations used throughout this study, it is important to place a caveat on the initial conditions prescribed to alternative cosmological models. Firstly, it is important to highlight that different cosmological observations constrain different parameters, such as the CMB constrains the amplitude of the matter power spectrum, and Supernovae measure the expansion rate over cosmic time at moderately low z. In addition, observational data are useful only in the context of some model, and the model may not be constrained if only one data set is used to constrain the parameters of that model. Given this, shared cosmological parameters such as the matter power spectrum normalization parameter $\sigma_g$ at z=0, although may be constrained within the context of \lcdm, may not be for alternative models. This allows some flexibility in setting cosmological parameters for the non-standard cosmologies \qcdm,\lwdm and CDE elucidated in section \ref{intro}. Specifically, \citet{Baldi_12} studies some coupled models and allocates shared cosmological parameters based on estimates derived from CMB observations through a \lcdm lens. Another option which we have opted for and was used by \citet{Carlesi_14a} is one in which we set both $\sigma_g$(z=0) and the matter density parameters based upon CMB observations by \textit{plank} interpreted via \lcdm model. Both options are valid, although the caveat in comparing the fiducial with the other cosmologies is that parameter differences may mask the sought out differences between cosmologies due to the physics and distinct evolution. 
\\
Following the methodology of \citep{Elahi_15} DARK-GADGET is the N-body code used to generate the simulations used throughout. It is a modified version of the P-GADGET code, the key modification is the inclusion of a separate gravity tree to account for additional long-range forces and an evolving DM-particle mass to allow for decay. For non-standard cosmologies, the code requires full evolution of the scalar field $\phi$, the mass of DM-particles and the expansion history. All cosmologies have cosmological parameters values (h,$\Omega_{b}$,$\Omega_{g}$,$\sigma_{g}$)=0.67,0.3175,0.049,0.83, making sure they are aligned with z=0, \lcdm Plank data \citep{Plank_14b,Plank_16}. In order to calculate the evolving linear power spectrum of the non-standard cosmologies along with the growth rate $\textit{f}\equiv dln D(a)/dln \textit{a} $, the publicly available CMBEASY \citep{Doran_05} code is used, a Boltzmann code to solve first-order Newtonian perturbation equations. 
\\
All cosmologies feature a 500$h^{-1}$Mpc sized box with $2 \times 512^{3}$ particles, where the particle mass at z=0 for all cosmologies are $m_{DM}(m_{gas})=6.9(1.3)\times 10^{10}h^{-1}M_{\odot}$, although the exact masses within the coupled cosmologies naturally depend on z. We look at 7 snapshots, beginning at z=2.98. Although, the simulations are initiated from z=100 with the same phases in their density perturbations, resulting in under-dense and over-dense regions in similar locations, thus LSS locations and quantities should also be similar. Initial conditions are produced using a uniform Cartesian grid along with the first-order Zel'Dovich approximation using an altered version of the publicly available N-GENIC code. The growth factors and expansion history calculated by CMBEASY are used to solve the particle displacements in the non-standard cosmologies. 

\begin{comment}
%for dm_only simulations, check Watts paper
We simulate cosmologies \lcdm \lwdm and \qcdm using a modified version of the GADGET-2 code\citep{Springel_05} named P-GADGET-2. This version is adapted to include non-standard dark sector physics, see \citet{Carlesi_14a} For more details. The three simulations to be compared are generated with the same initial phase information in order to give the study the best opportunity to distinguish the cosmic signatures of each cosmology. We also run the same cosmologies with various initial seeds as to test for cosmic variance. importantly, care must be taken in the context of conclusions to draw from the comparisons of non-standard cosmological simulations with the fiducial as in generating initial conditions there is some choice as to match observations at z=0 or $z_{CMB}$. 
\\
for all simulations, cosmological parameters are (h,$\Omega_{b}$,$\Omega_{g}$,$\sigma_{g}$)=0.67,0.3175,0.049,0.83, making sure all cosmologies are aligned with z=0, \lcdm Plank data \citep{Plank_14b,Plank_16}. The redshift of the CMB for the \qcdm is offset since the cosmological parameters stated are based upon the \lcdm model. This results in an early expansion history for \qcdm although results in the same $\sigma_{g}$ and expansion rate H(a=1) at z=0 as \lcdm. The growth rates are generated using CMBEASY and are utilized for generating the initial conditions in GADGET format, using the publicly available N-GENICI MPI code. 

%for dm_gas simulations, check Adremann paper
Following \citep{Elahi_15} We analyze three separate simulations with the following cosmological models: a reference \lcdm, an uncoupled scalar field representing DE model \qcdm and two coupled DM-DE cosmologies (CDE) with coupling parameters $\beta_{0}=0.05$ and 0.099, which are chosen to test the boundaries of allowed coupling and maximize differences between the fiducial cosmology. 
\\
The cosmological parameters for all simulations are (h,$\Omega_{b}$,$\Omega_{g}$,$\sigma_{g}$)=0.67,0.3175,0.049,0.83, making sure all cosmologies are aligned with z=0, \lcdm Plank data \citep{Plank_14b,Plank_16}. The number of particles are $512^{3}$ where the particle mass at z=0 for all cosmologies are $m_{DM}(m_{gas})=6.9(1.3)\times 10^{10}h^{-1}M_{\odot}$, although the exact masses within the coupled cosmologies naturally depend on z.
\end{comment}
\subsection{Large-Scale Structure Classification}\label{lssclass}

In order to classify the LSS for each cosmological simulation, the particles acquired from snapshots are used to generate a density field. We utilize the Delaunay Tessellation Field Estimator (DTFE)  in order to generate a suitable density field. Although the simplest method of generating a density field is by binning the positions of particles within a grid, this method of density field calculation leads to unphysical discontinuities and shot noise, which the DTFE method alleviates.
DTFE is an open source, c++ code \citep{Shaap_00,Weygaert_09,Cautun_11}. It is an adaptive method of density interpolation, in that it seeks out over-densities at the maximum possible resolution. It functions as follows;

1. DTFE creates Delaunay tetrahedra using the particle distribution, each tetrahedra connects 4 particles such that a circumscribing sphere around this tetrahedra will not intersect any other particles. 

2. Then, Voronoi cells are created from the tetrahedra, in which the density is then interpolated as a continuous field by using the volume of the cell along with the mass of each particle at its vertices. The cells vary in density linearly along the cell.
\begin{ceqn}
\begin{equation}
\textbf{\textit{H}}_{\alpha \beta}=\frac{\partial^{2}\rho(\textbf{x})}{\partial x_{\alpha}\partial x_{\beta}}\label{hess}
\end{equation}
\end{ceqn}
The DTFE density field is then used to form the Hessian Matrix, a technique which encapsulates the local curvature of a scalar field which can then be used to trace the LSS by taking the eigenpairs. The Hessian matrix is akin to the second derivative of a quadratic function, resulting in the maxima/minima of a parabola, but in 3 dimensional space. It is a second-order partial derivative of the density field, the 9 elements representing the 9 directions of 3-dimensional space (x-y,x-x,x-z...). 
\begin{ceqn}
\begin{equation}
G_{s}=\frac{1}{(2\pi\sigma^{2}_s)^{3}}e^{\Big(-\frac{(x^{2}+y^{2}+z^{2})}{2\sigma^{2}_{s}}\Big)}
\label{gaussian}
\end{equation}
\end{ceqn}
Prior to forming the Hessian matrix, smoothing the DTFE field is paramount in order to acquire a mask any left over density fluctuations which may lead to incorrect LSS classification. We convolve the raw density field $\rho(\textbf{x})$ with a 3-dimensional Gaussian distribution kernel $G_{s}$ (equation \ref{gaussian}) via the convolution theorem. This theorem states that the convolution of two functions $(f\ast g)$ is the inverse Fourier transform of the multiplication of f and g within Fourier space as per below:
\begin{ceqn}
\begin{equation}
\rho_{s}(\textbf{x})=\mathcal{F}^{-1}\Big\{\mathcal{F}(G_{s})\cdot \mathcal{F}(\rho(\textbf{x}))\Big\} 
\label{fourioreqn}
\end{equation}
\end{ceqn}
resulting from the convolution is the smoothed field $\rho_{s}(\textbf{x})$ with smoothing scale s. We probe our simulations at multiple scales ($\sigma_s$=2,3.5 and 5 Mpc/h) as increasing the smoothing scale allows us to investigate the alignment of halo spin with its environment for multiple larger scales. \citet{Trowland_13} demonstrates that tracking the evolution of alignments at multiple smoothing scales can give an insight into the evolution of LSS, from the various smoothing scales of spin-filament alignment it can be argued that filaments grow in size over time. 
\\
\\
Now we have the smoothed density field, we can complete the Hessian matrix. Importantly, we demonstrate the particular method in which we derived the Hessian matrix, we  first present the 3 dimensional Discrete Fourier Transform equation:
\begin{ceqn}
\begin{equation}
X_{\textbf{m}}=\sum_{\textbf{n}=0}^{\textbf{N}-1}x_{\textbf{n}}\cdot e^{-2\pi i(\textbf{mn}/\textbf{N})}
\label{dft}
\end{equation}
\end{ceqn}
Where $\textbf{n}=(n_{x},n_{y},n_{z})$ and $\textbf{m}=(m_{x},m_{y},m_{z})$ are vectors of grid points and frequencies, for each spatial dimension. We then take the $2^{nd}$ partial derivative of  $\mathcal{F}(\rho_{s}(\textbf{x}))$:

\begin{ceqn}
\begin{equation}
\frac{\partial^{2} \mathcal{F}(\rho_{s}(\textbf{x}))}{\partial n_{\alpha} \partial n_{\beta}}=\frac{-4\pi^{2}m_{\alpha}m_{\beta}}{N^{2}}\sum_{\textbf{n}=0}^{\textbf{N}-1}\rho_{s(\textbf{n})}e^{-2\pi i(\textbf{mn}/\textbf{N})} 
\label{dftdiff}
\end{equation}
\end{ceqn}

Where $\alpha$,$\beta$=x,y,z and the wavenumber, defined as $k=\frac{-4\pi^{2}m_{\alpha}m_{\beta}}{N^{2}}$ is multiplied throughout the Hessian Matrix elements, with $m_{\alpha}m_{\beta}$ determining the orientation of the fourier-space grid. Once the 6 unique elements of the Hessian have been calculated, we can take the inverse Fourier transform:
\begin{ceqn}
\begin{equation}
\textbf{\textit{H}}_{\alpha \beta}=\mathcal{F}^{-1}\Big\{k\cdot F_{m_{\alpha \beta}}\Big\}
\label{invfouriordtf}
\end{equation}
\end{ceqn}
It is of great importance for the Hessian matrix to be formed within Fourier space as such, in order to avoid numerical effects resulting in inaccurate eigenvectors.

\begin{table}
\centering
\begin{tabular}{|c|c|c|c|c|}
	\hline
	LSS Type & $\lambda_{1}$ & $\lambda_{2}$ & $\lambda_{3}$ & LSS axis\\
	\hline
    Cluster & $<0$ & $<0$ & $<0$ & $\textbf{e}_{3}$\\
	\hline
    Filament & $<0$ & $<0$ & $>0$ & $\textbf{e}_{3}$\\
	\hline
    Sheet & $<0$ & $>0$ & $>0$ & $\textbf{e}_{3}$\\
	\hline
    Void & $>0$ & $>0$ & $>0$ & $\textbf{e}_{3}$\\
	\hline   
    
\end{tabular}
\caption{\label{eigpairs}Depending on the eigenvalues and their values, we classify each voxel as a Cluster, Filament, Sheet or Void. The LSS axis is represented by the last collapse direction, being $\textbf{e}_{3}$ regardless of the LSS type.}
\end{table}

Mapping the Hessian matrix via taking the eigenpairs for each voxel allows us to encapsulate the curvature information along the 3 spatial dimensions within the eigenvalues and eigenvectors.
Using the eigenvalues $\lambda_{1}$,$\lambda_{2}$,$\lambda_{3}$ we can classify each voxel with one of 4 LSS, as per Table \ref{eigpairs}. The corresponding eigenvectors $\textbf{e}_{1}$,$\textbf{e}_{2}$,$\textbf{e}_{3}$ represent the directions associated with their eigenvalues. The eigenvector corresponding with the smallest eigenvalue (represented by $\textbf{e}_{3}$)  pints in the direction of slowest collapse (see Section \ref{intro} for details), thus will be used to represent the axes of all LSS.

\subsection{Spin-LSS Alignment}\label{alignment}

\subsubsection{Halo Classification}\label{halo_spin}
The very same particles which are used to classify LSS are then used to feed into VELOCIraptor, a Dark Matter Halo classifier (A.K.A STructure Finder) \citep{Elahi_11}. VELOCIraptor is a sub(halo) finder which works in a two-step process:
1. Halos are first identified using a  3DFoF algorithm pruned for any artificial particle bridges using a 6DFoF and the velocity dispersion of the FoF group.
2.Then substructure is found by inspecting the dynamics, such as velocity distribution of particles and their distinctness with the halo environment. The substructure is linked via a phase-space FoF algorithm. 

VELOCIraptor also generates the spins of the catalog halos, but we filter out halos with at least 100 particles as the spin measurements are more reliable at this threshold. 
\begin{ceqn}
\begin{equation}
\textbf{J}=\sum_{i=0}^{N}\textbf{r}_i\times m_i\textbf{v}_i \label{AM}
\end{equation}
\end{ceqn}

The spin of each halo is calculated by including all the associated particles (N) for each halo, as per equation \ref{AM}. $\textbf{r}_i,m_i and \textbf{v}_i$ are the particle radius from halo centre, mass and velocity, respectively.

\subsubsection{dot product}

Given the spin of halos (section \ref{halo_spin}) and the axes of the LSS (section \ref{lssclass}) we take the dot product of the spin \textbf{J} with the LSS axes vector $\textbf{e}_{3}$. As detailed in section \ref{intro}, $\textbf{e}_{3}$ aligns with the slowest collapse direction, thus it is an ideal definition for the direction of all LSS which we use for the spin-filament dot product alignment measure, see above equation.
Since all we require is the alignment, we take the absolute value of the dot products thus we are left with the alignment from 0 to 90 degrees. 
\begin{ceqn}
\begin{equation}
cos(\theta)=\bigg|\frac{\textbf{J}\cdot \textbf{e}_{3}}{|\textbf{J}| |\textbf{e}_{3}|}\bigg|\label{dotprod}
\end{equation}
\end{ceqn}

\subsubsection{Alignment Correlations} 

Equation \ref{mod} is the model which was derived by \citet{Lee_11}, that takes into account the unit spin vector of structures and the tidal field in order to quantify the degree of alignment. It is used in this study to fit to the distribution of the alignment between halo spin and $\textbf{e}_{3}$. The model is a good indication to show the strength of alignment (based on the correlation coefficient c and whether it aligns with the theory of TTT in that it is predicted spin shall be parallel aligned thus the correlation coefficient shall be negative) to the tidal field. but as introduced, the spin may also become orthogonal, especially of high mass halos, due to mergers and accretion and perhaps a later turn around before the halos becomes virialized.
\begin{ceqn}
\begin{equation}
\textit{P}(cos\theta)=(1-c)\sqrt{1+\frac{c}{2}}\Big[1-c\Big(1-\frac{3}{2}cos^{2}\theta\Big)\Big]^{-3/2} \label{mod}
\end{equation}
\end{ceqn}

The more popular method in portraying the alignment signals is via taking the mean/median of the dot product values for each mass bin, and calculating the error of the mean/median using Bootstrap re-sampling. Although this method is more popular, this study will fit to the model as it gives a good indication of the strength of alignment but also because it gives a good indication of how the results compare to TTT.

\section{Results}\label{results}
 
\begin{figure}
\centering
\includegraphics[width=0.5\textwidth]{dm_only_cde0_snapshot_011_DTFE_ALIGNMENT_PLOT_gd1250_smth2Mpc.png}
\label{cor_fig} 
\end{figure}
\begin{figure}
\centering
\includegraphics[width=0.5\textwidth]{dm_only_cde0_snapshot_011_DTFE_bootstrappost_LSS2_spin_sim500Mpc_grid1250_smth2Mpc_4bins_partclfilt100_.png}\label{cor_fig_cde0} 
\end{figure}
\begin{figure}
\centering
\includegraphics[width=0.5\textwidth]{dm_only_cde0_snapshot_011_DTFE_grid_MOD_FIT_PLOT_gd1250_smth_smth2Mpc.png}\label{cor_fig_wdm2} 
\caption{PRELIMINARY}
\end{figure}
\begin{figure}
\centering
\includegraphics[width=0.5\textwidth]{dm_only_cde0_snapshot_011_DTFE_grid_MOD_FIT_PLOT_gd1250_smth_smth2Mpc_dp_mthd_increm_overlap90perc.png}\label{cor_fig_wdm2} 
\caption{PRELIMINARY}
\end{figure}
So I could start with just the simple plots I have on here of just alignment of halo with filament, use this as a simple first demonstration. Then move on to showing the same alignment in median method, noting that it is a common way to portray alignment. Then write about the moving bin and place in the larger plots I have to show residuals etc. and speak about my decision to involve the halo count as halo numbers severely reduce which could affect things. But I can also take not of all the trends I see such as smoothing scale trends and redshift trends and coordinate with Holly's papers and others who find it. Then put in the velocity although they may be useless and write about seeing perhaps different type of bulk flows etc. Then include the spins of all LSS.
\\


Figure 1 shows that that there is minor differences between alignment signals for each of the cosmologies considered. The slight dip within the second mass bin from the left side for the non-standard cosmologies may be worth further pursuit although they both lie within the error range of \citet{Trowland_13} signal. We also include these correlations for 1,2 and 5 Mpc/h

\begin{figure*}
\centering
\includegraphics[width=0.7\textwidth]{dm_only_cde0_snapshot_011_DTFE_ALIGNMENT_PLOT_gd1250_smth2Mpc.png}
\label{comb_fig} 
\caption{This figure shows overplotting of 3 cosmologies as in legend for 3 smoothing scales}
\end{figure*}

Figure \ref{comb_fig} shows that cosmologies with the same initial seeds slightly differ as a function of smoothing scale. From the figure it is seen that the most difference seems to be at 3.5Mpc/h scales but there is also some statistically significant difference within the lowest mass halos within the 2Mpc/h smoothing scales. The contrary is true at larger smoothing scales since smaller scale features are smoothed out and result in very similar comparisons for all cosmologies. Although, we find that there are discrepancies within the high mass bin of high smoothing scales.
\\



\section{Discussion}\label{discussion}



\section{Conclusion}\label{conclusion}

\bibliographystyle{mnras}
\bibliography{biblio} 

% Don't change these lines
\bsp	% typesetting comment
\label{lastpage}
\end{document}

% End of mnras_template.tex
