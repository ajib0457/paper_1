% mnras_template.tex
%
% LaTeX template for creating an MNRAS paper
%
% v3.0 released 14 May 2015
% (version numbers match those of mnras.cls)
%
% Copyright (C) Royal Astronomical Society 2015
% Authors:
% Absem Jibrail (The University of Sydney)

% Change log
%
% v3.0 May 2015
%    Renamed to match the new package name
%    Version number matches mnras.cls
%    A few minor tweaks to wording
% v1.0 September 2013
%    Beta testing only - never publicly released
%    First version: a simple (ish) template for creating an MNRAS paper

%%%%%%%%%%%%%%%%%%%%%%%%%%%%%%%%%%%%%%%%%%%%%%%%%%
% Basic setup. Most papers should leave these options alone.
\documentclass[a4paper,fleqn,usenatbib]{mnras}

% MNRAS is set in Times font. If you don't have this installed (most LaTeX
% installations will be fine) or prefer the old Computer Modern fonts, comment
% out the following line
\usepackage{newtxtext,newtxmath}
% Depending on your LaTeX fonts installation, you might get better results with one of these:
%\usepackage{mathptmx}
%\usepackage{txfonts}

% Use vector fonts, so it zooms properly in on-screen viewing software
% Don't change these lines unless you know what you are doing
\usepackage[T1]{fontenc}
\usepackage{ae,aecompl}

%%%%% AUTHORS - PLACE YOUR OWN PACKAGES HERE %%%%%

% Only include extra packages if you really need them. Common packages are:
\usepackage{graphicx}	% Including figure files
\usepackage{amsmath}	% Advanced maths commands
\usepackage{nccmath}
\usepackage{amssymb}	% Extra maths symbols
\usepackage{verbatim} % To allow for commenting out of blocks of text using comment
\usepackage{subcaption}
\captionsetup{compatibility=false}

%%%%%%%%%%%%%%%%%%%%%%%%%%%%%%%%%%%%%%%%%%%%%%%%%%

%%%%% AUTHORS - PLACE YOUR OWN COMMANDS HERE %%%%%

% user defined commands
% units
\def \Msun{\ {\rm M_\odot}}
\def \Msunh{\ h^{-1}{\rm M_\odot}}
\def \Mpc{{\rm Mpc}}
\def \kpc{{\rm kpc}}
\def \Mpch{\ h^{-1}{\rm Mpc}}
\def \kpch{\ h^{-1}{\rm kpc}}
\def \Gpch{\ h^{-1}{\rm Gpc}}
\def \lcdm{$\Lambda$CDM}
\def \lwdm{$\Lambda$WDM}
\def \qcdm{$\phi$CDM}
\newcommand{\dedm}[1]{$\phi(\beta_o=#1)$CDM}

% reference commands
\newcommand{\Eqref}[1]{Eq.~(\ref{#1})}
\newcommand{\Figref}[1]{Fig.~\ref{#1}}
\newcommand{\Secref}[1]{\S\ref{#1}}  
\newcommand{\Tableref}[1]{Table~\ref{#1}}


% for emphasizing comments
%\newcommand{\PJE}[1]{{\bf\color{Red}PJE-{#1}}}

% number of cosmologies
\newcommand{\ncosmo}{3}

% Please keep new commands to a minimum, and use \newcommand not \def to avoid
% overwriting existing commands. Example:
%\newcommand{\pcm}{\,cm$^{-2}$}	% per cm-squared


%%%%%%%%%%%%%%%%%%%%%%%%%%%%%%%%%%%%%%%%%%%%%%%%%%

%%%%%%%%%%%%%%%%%%% TITLE PAGE %%%%%%%%%%%%%%%%%%%

% Title of the paper, and the short title which is used in the headers.
% Keep the title short and informative.
\title[\textit{N}-Body Studies of Non-Standard Cosmologies]{ \textit{N}-Body Studies of Non-Standard Cosmologies: Spin Correlations and Large-Scale Structures }

% The list of authors, and the short list which is used in the headers.
% If you need two or more lines of authors, add an extra line using \newauthor
\author[A. W. Jibrail et al.]{
Absem W. Jibrail,$^{1}$\thanks{E-mail: ajib0457@uni.sydney.edu.au (EA)}
Geraint F. Lewis$^{1}$
and Pascal J. Elahi$^{1,2}$
\\
% List of institutions
$^{1}$Sydney Institute for Astronomy, School of Physics, A28, The University of Sydney, NSW, 2006, Australia\\
$^{2}$International Centre for Radio Astronomy Research (ICRAR), The University of Western Australia, 35 Stirling Hwy, \\
Crawley, Western Australia 6009, Australia}

% These dates will be filled out by the publisher
\date{Accepted XXX. Received YYY; in original form ZZZ}

% Enter the current year, for the copyright statements etc.
\pubyear{2018}

% Don't change these lines
\begin{document}
\label{firstpage}
\pagerange{\pageref{firstpage}--\pageref{lastpage}}
\maketitle

% Abstract of the paper
\begin{abstract}
Despite the Standard Cosmological Model's ($\Lambda$CDM) predictive accuracy on the largest scales of the universe, there are conceptual and observational shortfalls on galactic scales which loom large within dark sector cosmology. Problems of \textit{fine-tuning} the vacuum energy density $\Lambda$, a Dark Energy (DE)- Dark Matter (DM) \textit{coincidence} problem and the Missing Satellite problem, have inspired theorists to posit and study - the latter using $N$-body simulations - non-standard cosmologies in an effort to alleviate such shortfalls. This is a comparative study between a Warm Dark Matter (\lwdm), an uncoupled Quintessence (\qcdm) along with the fiducial \lcdm, conducted using the evolution of Angular Momentum as a probe.  

%This is a simple template for authors to write new MNRAS papers. The abstract should briefly describe the aims, methods, and main results of the paper. It should be a single paragraph not more than 250 words (200 words for Letters). No references should appear in the abstract.
\end{abstract}

% Select between one and six entries from the list of approved keywords.
% Don't make up new ones.
\begin{keywords}
cosmology: simulations -- cosmology: large-scale structure of the universe -- dark matter -- dark energy
\end{keywords}

%%%%%%%%%%%%%%%%%%%%%%%%%%%%%%%%%%%%%%%%%%%%%%%%%%

%%%%%%%%%%%%%%%%% BODY OF PAPER %%%%%%%%%%%%%%%%%%

\section{Introduction}
\subsection{broad structure formation introduction}
The Cosmic Microwave Background (CMB) shows an anisotropic temperature distribution within the early universe which is thought to have occurred via quantum fluctuations during the period of Inflation \citep{Guth_82}, causing minor inhomogeneities that cascaded post-inflation through the process of gravitational instability into what we see today as knots, filaments, sheets and voids. According to linear theory and the Zel'dovich approximation \citep{Zeldovich_70} evolution of the cosmic web seemed to have begun from over-dense regions which collapsed along their shortest axes into sheets, then from sheets to filaments and finally into knots. This large-scale bulk flow provides the tidal field unto which DM halos form and gain their initial spin (spin refers to the halo unit vector direction of Angular Momentum hereafter) and Tidal Torque Theory (TTT) describes the procurement of the initial Angular Momentum of dark matter halos, that is through the influence of the tidal field, which has a great influence on the progenitor before turn-around \citep[Holye 1949][]{Peebles_69,Zeldovich_70} 
It is argued that since the tidal field is the manifestation of the cosmological environment, and this field predominantly, although arguably, is responsible for the acquisition of initial galactic spins, it is conjectured the imprint of cosmology is preserved within the spins of galaxies \citep{Lee_pen_00} and thus may be a suitable probe of cosmology. In contrast, it has been shown that since over-dense regions such as sheets, filaments and clusters are susceptible to non-linear physics, which possibly erases or masks the cosmological signatures that could otherwise serve to distinguish between cosmologies.

\subsection{spin research overview}
\subsubsection{simulations}
There are numerous studies which investigate the large scale structure (LSS) of \lcdm cosmology by using spin alignment and evolution as a probe. Studies such as \citet{Faltenbacher_02,Calvo_07,Zhang_09} and \citet{Hahn_07} find that there are clear correlation signals between LSS axes and halo spins, which can be linked to the formation of LSS and its influence on the spin and shapes of halos. Specifically, \citet{Zhang_09} find that the spin and shape of halos within filaments of M $\leq$ $10^{13}$ $h^{-1}$ \(M_\odot\) are aligned with the filament axes. Furthermore they find that the alignments are strengthened for halos which are closes to more massive node (cluster) halos. They interpret this as 'transit large-scale environment impact' that is a transit from two-dimensional filaments to three-dimensional clusters.

\citet{Trowland_13} conducts a spin evolution study using the Millennium simulations and finds that at low redshift, low mass DM halo spins ( $\sim$$10^{11.6}$ to $10^{12.9}$ $h^{-1}$ \(M_\odot\)) tend to be aligned to their filaments whereas high mass halos ( $\sim$ $10^{12.9}$ to $10^{13.4}$ $h^{-1}$ \(M_\odot\)) tend to be orthogonal in their alignment, suggesting that TTT may not be the sole mechanism for spin acquisition.  

The alignment of halo spin with LSS is not exclusive to filamentary structure, but is also seen within sheets and clusters, where there is a general alignment between halo spin and the e2 (initial intermediate tidal tensor) axes, also known as the axes of slowest collapse. \citep{Libeskind_12,Dubois_14,Calvo_14,Kang_15,Wang_17,Veena_18} 

\citet{Bond_96,Codis_12} pichon et al. 2015 tell the story of how the spin evolution of halos are part in parcel of the pancaking effect whereby the fastest collapsing axis e3 causes mergers and accretion along this axis and explaining the alignments found. The horizon simulations \citep{Dubois_14,Welker_14} also show evidence of bulk flow as does \citep{Trowland_13} which explains the spin flip of high mass halos by undergoing major mergers, whereas low mass halos being less likely to undergo mergers retain their parallel spin to the slowest collapsing axis.

\subsubsection{observations}
These alignments have not only been found in simulation but also in observation \citep{Jones_10}. \citet{Pen_00} also shows tentative alignment in spirals. 
\citet{Lee_Erdogdu_07} show that galaxies in the Tully-Fisher catalog are weakly orthogonal with their environment, stating an average correlation coefficient (c) value of$\bar{c}=0.084\pm 0.014$ (where c=0 represents a random alignment) to their LSS with 99.9$\%$ confidence (based on Kolmogorov-Smirnov statistic) that the null hypothesis of no spin-shear correlation is rejected. 
\\
Direct measurements of spin are also being conducted with Integrated Field units (IFU), such surveys as SAMI \citep{Croom_12} Thus DM halo spin could be a useful tool to probe the universe and find discrepancies with \lcdm.

\subsection{alternative N-body sims probes of cosmology}
There have been many studies which compare Non-Standard Cosmologies with various probes. From studies such as \citet{Elahi_14} finds that above the suppression scale, on the scales of galaxy clusters, there are marked differences between $\Lambda CDM$ and WDM. Whereas \citet{Elahi_15} shows little to no difference between $\Lambda CDM$ and the coupled cosmologies. \citet{Adermann_17} finds that there is a potentially observable difference between the volume distribution of voids between $\Lambda CDM$ and the CDE (coupled DM-DE) cosmology. Also, there are differences between the densities of voids for each cosmology, being emptiest in uncoupled quintessence model and densest within $\Lambda CDM$. \citet{Carlesi_14a} finds using hydrodynamical simulations that their  self interacting quintessence model uDE provides a higher concentration of halos within the LSS as compared to their fiducial $\Lambda CDM$ and other cosmologies compared. \citet{Carlesi_14b} find a weak coupling between the spin, triaxiality and virialisation and the cosmology dark sector types. \citet{Smith_11} find that WDM model suppresses the halo mass function by 50$\%$ for masses 100 times the free-streaming mass scale. \citet{Watts_17} finds there are higher cluster abundances and lower void abundances within the quintessence cosmology with respect to $\Lambda CDM$. Also speculates that DM with gas simulations may give more pronounced differences in the discrepancies.

\subsection{introduction of $\Lambda$CDM and its shortfalls}

The \lcdm model does exceptionally well to predict large-scale observations such as the anisotropies in the Cosmic Microwave Background (CMB) \citep{Bennett_13,Plank_14b,Plank_16}, features in the LSS \citep{Abazajian_09}, Baryonic Acoustic Oscillations (BAO) \citep{Beutler_11} and weak lensing \citep{Kilbinger_13}.
Despite all of its success, \lcdm suffers from various observational inaccuracies as well as conceptual shortfalls regarding the dark sector physics (DM and DE). although constrains from observations have supported the CDM scenario \citep{Bertone_05,Petraki_13} and having well-motivated candidates from particle physics over more energetic dark matter such as Hot Dark Matter. DE in the form of a cosmological constant seems to work well to predict the late time accelerated expansion of the universe \citep{Suzuki_12}

\subsection{approaches to dark sector research}
There are active fields of research which hunt for dark matter particles, such as Weakly-Interacting Massive Particles (WIMPS), or explain away dark matter in the form of Massive Compact halo Objects (MACHOS) \citep{Alcock_00} or Modified Newtonian Dynamics (MOND) \citep{Milgrom_15}, and other Dark Matter and Dark energy alternatives \citep[e.g][]{Mannheim_06}. The remaining prong of research posits various non-standard cosmologies with distinct characteristics inferred by Lambda-CDM of the dark sector such as a cosmological constant and non-relativistic Dark Matter. 
This includes Warm Dark Matter, which poses to alleviate the Missing-Satellite problem, where ($\Lambda$CDM) produces too many satellite around central galaxies \citep{Klypin_99,Moore_99}. although some have proposed that this is not an issue of the standard cosmological model but rather it is to do with the limitations of Dark matter simulations \citep{Wetzel_16} or other feedback \citep[e.g][]{Bullock_00}. Although hydrodynamical simulations have also been criticized for their poor representations.
Another is a quintessence model which poses to alleviate the \textit{fine-tuning} of the initial value for vacuum energy density.
(during statements for the largest issues of ($\Lambda$CDM), introduce posited non-standard cosmos in order to alleviate issues)
A conceptual problem with ($\Lambda$CDM) is the coincidence problem:  the values of energy densities of DM and DE are similar at present day, which seems quite unlikely given the age of the universe, thus perhaps there may be some inter-dependence within the dark sector which alleviates the otherwise unlikely coincidence. 

\subsection{details of remainder of paper}
Section \ref{cosmos} of this paper will introduce the main components of the Non-Standard Cosmologies considered within this study. Section \ref{method} is the methodology which introduces the simulations and structure classifications. This is followed by Section \ref{results} where the main findings will be presented, then Section \ref{discussion} contains the discussion of the results and Section \ref{conclusion} summarizes and concludes the findings. 

\section{Non-Standard Cosmological Models}\label{cosmos}
The cosmological models to be compared tweak the dark sector of the standard model, specifically, Dark Matter in the form of Warm Dark Matter and Dark Energy in the form of Uncoupled Quintessence. 

\subsection{Uncoupled Quintessence  (\qcdm)}
The general form of the Langrangian which describes the scalar field of \qcdm model
\begin{ceqn}
\begin{equation}
L=\int d^{4}x\sqrt[]{-g}\big(-\frac{1}{2}\partial_{\mu}\partial^{\mu}\phi+V(\phi)+m(\phi)\psi_{m}\bar{\psi}_{m}\big)
\end{equation}
\end{ceqn}
includes a kinetic term, potential term $V(\phi)$ and an interaction term $(\psi_{m})$ with Dark Matter.

The \qcdm model includes no direct interaction between Dark Matter and the quintessence field, thus $m(\phi)=m_{0}$. But Dark Energy - usually represented by a cosmological constant $\lambda$ - is replaced with a time-dependent, evolving scalar field whereby regions of the universe have the opportunity to expand independently, which alleviates the cosmological constant problem \citep{Joyce_15}.
\begin{ceqn}
\begin{equation}
V(\phi)=V_{0}\phi^{-\alpha}\label{rp}
\end{equation}
\end{ceqn}
As for the potential term $V(\phi)$, this cosmology uses the Ratra-Peebles Potential  \citep{Ratra_88},where $V_{0}$ and $\alpha$ are constants which are fit through observational data, and $\phi$ is in units of Plank mass. This potential term is a contrived field potential such that at late times the quintessence field dominates the energy budget of the universe, and at early times it 'tracks' the energy density of matter and radiation akin to observations \citep{Joyce_15}.

\subsection{Warm Dark Matter  ($\Lambda$WDM)} 
The $\Lambda$CDM is extremely successful on large scales, thus WDM aims to merely modify the $\Lambda$CDM model in a effort to damp small-scale structure of the universe. WDM features Dark Matter particles which move at relativistic velocities, increasing the length of the free-streaming of particles, which will smooth out over-densities and suppress structure formation at scales smaller than the co-moving scale\citep{Bode_01}.
The free-streaming limit of the WDM particle has been confined via observations of the Lyman-alpha forest, to a lower limit mass $m_{WDM}\gtrapprox$ 3.3KeV. \citep{Viel_13} Despite this lower limit, the mass we assign to our WDM model has particle energy = 2keV to exaggerate the cosmological effects. INCLUDE DETAILS ABOUT SCALE OF STRUCURE FORMATION AS PER ANDREWS PAPER  

\section{Methodology}\label{SiriusBlack}\label{method}

% Example table
\begin{table}
	\centering
	\caption{Cosmological parameters for reference cosmology ($\Lambda$CDM) as well as WMD \& \qcdm}
	\label{tab:1}
	\begin{tabular}{lccr} % four columns, alignment for each
		\hline
		Model & $\Omega_{m}$ & $\Omega_{b}$ & $\sigma_{g}$  \\
		\hline
		$\Lambda$CDM(ref.) & 0.3175 & 0.049 & 0.83\\
		WDM & $512^{3}$ & 8.16$\times$$10^{10}$ & 8 \\
		\qcdm & $512^{3}$ & 8.16$\times$$10^{10}$ & 9 \\
		\hline
	\end{tabular}
\end{table}

\subsection{\textit{N}-Body Simulations}
Following \citep{Elahi_15} We produced three separate simulations with the aforementioned cosmological models. The cosmological parameters for all simulations are (h,$\Omega_{b}$,$\Omega_{g}$,$\sigma_{g}$)=0.67,0.3175,0.049,0.83, making sure all cosmologies are aligned with z=0, $\Lambda$CDM Plank data \citep{Plank_14b,Plank_16}. The WDM model  simulations used to do this research are Pascal’s Dark matter only simulations run with the Gadget code.(incl. reference, and summarize how these simulations were run including which code are they based upon etc. then state ‘for more information, see Elahi et al.) which are DM-only simulations of 500Mpch for all cosmologies that will be compared during this study. The number of particles are $512^{3}$ where the particle mass is 1010Msol table~\ref{tab:1}

\subsection{Quantifying the Large-Scale Structure}

The snapshots of the simulation particles at z=0 are used to generate a density field: A density field is required in order to classify the LSS and attain the LSS axes, thus we utilize  the Delaunay Tessellation Field Estimator (DTFE)  in order to generate a suitable density field. 
The simplest technique to generate a density field from a snapshot of particles is via the binning method: that is breaking up the simulation box space into a 3 dimensional grid in which to bin each particle and prescribe a value in each voxel relative to its mass.

Alas, this method of density field calculation leads to unphysical discontinuities and shot noise, which the DTFE method alleviates.
DTFE method is an open source code, c++ code \citep{Shaap_00,Weygaert_09,Cautun_11}. It is an adaptive method of density interpolation in that it seeks out the over-densities at the maximum possible resolution. The way it functions is as follows;

1. Firstly, it creates Delaunay tetrahedra using the particle distribution, each tetrahedra connects 4 particles such that a circumscribing sphere around this tetrahedra will not intersect any other particles. 

2. Then, Voronoi cells are created from the tetrahedra, in which the density is then interpolated as a continuous field by using the volume of the cell along with the mass of each particle at its vertices. The cells vary in density linearly along the cell.
\begin{ceqn}
\begin{equation}
\textbf{\textit{H}}_{\alpha \beta}=\frac{\partial^{2}\rho_{s}(\textbf{x})}{\partial x_{\alpha}\partial x_{\beta}}\label{hess}
\end{equation}
\end{ceqn}
The density field can then be processed through what is known as the Hessian Method in order to classify the Large-scale structure. Analytically, the Hessian Matrix can be attained by taking the second derivative of the smoothed density field in every direction for each voxel and ending up with a 5th-order tensor. This is not unlike the quadratic whereby the second derivative is taken in order to find the maxima/minima of the curve.
Firstly, smoothing the DTFE field is paramount in order to acquire a continuous representation of the simulation and to avoid mis-classification of the Large-Scale structure. We convolve the density field with a normal distribution kernel via the convolution theorem. The convolution of two functions $(f\ast g)$ is the inverse Fourier transform of the multiplication of f and g within Fourier space.
\begin{ceqn}
\begin{equation}
\rho_{s}(\textbf{x})=\mathcal{F}^{-1}\Big\{\mathcal{F}(G_{s})\cdot \mathcal{F}(\rho(y))\Big\} 
\end{equation}
\end{ceqn}
where $\rho(y)$ is the raw density field from DTFE and resulting from the convolution is $\rho_{s}(\textbf{x})$ the smoothed field with smoothing scale s Mpc/h from the Gaussian $G_{s}$
\begin{ceqn}
\begin{equation}
G_{s}=\frac{1}{(2\pi\sigma^{2}_s)^{3/2} }e^{\Big(-\frac{(y-x)^{2}}{2\sigma^{2}_{s}}\Big)}
\end{equation}
\end{ceqn}
To retrieve the Hessian matrix from the smoothed density field, take the 2nd partial derivative of the Discrete Fourier Transform (DFT) of the smoothed density field:
\begin{ceqn}
\begin{equation}
\frac{\partial^{2} (F_{m})}{\partial n_{\alpha} \partial n_{\beta}}=\frac{-4\pi^{2}m^{2}}{N^{2}}\sum_{n=0}^{N-1}\rho_{s(n)}(\textbf{x})e^{-2\pi i(nm/N)} 
\end{equation}
\end{ceqn}
for n=0,1,2...n-1 is the sample of points and N=0,1,2...N-1 is the frequency. The wavenumber $k=\frac{-4\pi^{2}m^{2}}{N^{2}}$ is a constant in this case which is multiplied to the smoothed density field in all dimensions (x,y,z) in Fourier space.
\begin{ceqn}
\begin{equation}
\textbf{\textit{H}}_{\alpha \beta}=\mathcal{F}^{-1}\Big\{k\cdot F_{m}\Big\}
\end{equation}
\end{ceqn}
There are 6 unique components to produce the 9 elements of the Hessian of equation \ref{hess}, where $\alpha$,$\beta$=x,y,z. It is of great importance that the Hessian matrix is formed within Fourier space in order to avoid numerical effects and inaccurate filament axes.

\subsection{Alignment of halo spin}

Taking the eigenpairs of the $4^{th}$ order tensor which results from applying the Hessian method to each voxel of the density field allows us characterize each voxel within the field as either a:
\\
\\
\textbf{Cluster} $\lambda_{1}$<0, $\lambda_{2}$<0, $\lambda_{3}$<0
\\
\textbf{Filament} $\lambda_{1}$<0, $\lambda_{2}$<0, $\lambda_{3}$>0
\\
\textbf{Sheet} $\lambda_{1}$<0, $\lambda_{2}$>0, $\lambda_{3}$>0
\\
\textbf{Void} $\lambda_{1}$>0, $\lambda_{2}$>0, $\lambda_{3}$>0
\\

where each voxel features $\lambda_{1}$,$\lambda_{2}$,$\lambda_{3}$ and the corresponding eigenvectors $\textbf{e}_{1}$,$\textbf{e}_{2}$,$\textbf{e}_{3}$ represent the directions associated with their eigenvalues. As the eigenvalues are ordered by size, the smallest eigenvalue corresponds with $\textbf{e}_{3}$  which itself represents the slowest collapsing axes, thus will be used to represent the filaments axes direction thereafter.

The simulations produce snapshots which are then used to feed into VELOCIraptor, a Dark Matter Halo classifier (A.K.A STructure Finder) \citep{Elahi_11}. VELOCIraptor is a sub(halo) finder which works in a two-step process:
1. Halos are first identified using a  3DFoF algorithm and artificial  particle bridges are cut off via a 6DFoF algorithm, as well through the velocity dispersion of the FoF group.
2.Then follows the convention of treating smaller objects as sub-halos and larger objects as host halos. 
A linking length of 0.1 is used

Only halos classified with more than 100 particles or more are taken into account during this study as halos spin measurements are more reliable at this threshold. 
\begin{ceqn}
\begin{equation}
\textbf{J}=\sum_{i=0}^{N}\textbf{r}_i\times m_i\textbf{v}_i \label{AM}
\end{equation}
\end{ceqn}
The spin of each halo is calculated by taking all the particles involved and the cross-product is taken between the radius from halo center to particle center and the mass multiplied by the velocity of the particle, then sum up all of those for each particle within a halo.
\begin{ceqn}
\begin{equation}
cos(\theta)=\bigg|\frac{\textbf{J}\cdot \textbf{e}_{3}}{|\textbf{J}| |\textbf{e}_{3}|}\bigg|\label{dotprod}
\end{equation}
\end{ceqn}
Given the spin of halos and the axes of the LSS we take the dot product of the unit vector of the spin \textbf{J} with the LSS axes unit vector $\textbf{e}_{3}$.
Since all we require is the alignment, we take the absolute value of the dot products thus we are left with the alignment from 0 to 90 degrees. 

THINK ABOUT WHETHER TO FIT CORRELATIONS TO MODEL OR USE MEDIAN \& BOOTSTRAP RESAMPLING
\subsubsection{Model Fitting} 
\begin{ceqn}
\begin{equation}
\textit{P}(cos\theta)=(1-c)\sqrt{1+\frac{c}{2}}\Big[1-c\Big(1-\frac{3}{2}cos^{2}\theta\Big)\Big]^{-3/2} \label{mod}
\end{equation}
\end{ceqn}
Equation \ref{mod} is the model which was derived by \citet{Lee_11}, is used to fit to the distribution of the alignment between halo spin and the minor axis of the tidal field, that is the filament axis. The model is a good indication to show the strength of alignment (based on the correlation coefficient c and whether it aligns with the theory of TTT in that it is predicted spin shall be parallel aligned (thus the correlation coefficient shall be negative) to the tidal field. but as introduced, the spin may also become orthogonal, especially of high mass halos, due to mergers and accretion and perhaps a later turn around before the halos becomes virialized.

The error is calculated via MCMC? The grid method seems fine so this is probably not needed since the model fitting is too simple.
\subsubsection{Median Method}
The more popular method in portraying the alignment signals is via taking the mean/median of the dot product values for each mass bin, and calculating the error of the mean/median using Bootstrap re-sampling. Although this method is more popular, this study will fit to the model as it gives a good indication of the strength of alignment but also because it gives a good indication of how the results compare to TTT.

There are many techniques in comparing two different signals, including cross-correlating using the auto-correlation algorithm, even taking the residuals can be a satisfactory approach to compare two alignment signals.

\section{Results}\label{results}
 
\begin{figure}
\centering
\includegraphics[width=0.5\textwidth]{MOD_FIT_PLOT_grid_850_smth_scl3_5.png}
\label{cor_fig} 
\end{figure}
\begin{figure}
\centering
\includegraphics[width=0.5\textwidth]{MOD_FIT_PLOT_grid_850_smth_scl3_5_cde0.png}\label{cor_fig_cde0} 

\end{figure}
\begin{figure}
\centering
\includegraphics[width=0.5\textwidth]{MOD_FIT_PLOT_grid_850_smth_scl3_5_wdm2.png}\label{cor_fig_wdm2} 
\caption{PRELIMINARY: The top plot shows the strength of alignment as measured via the correlation coefficient from fitting equation \ref{mod} using mcmc. for $\Lambda$CDM. Middle plot shows the same but for \qcdm and the bottom plot shows for WDM}
\end{figure}
Figure 1 shows that that there is minor differences between alignment signals for each of the cosmologies considered. The slight dip within the second mass bin from the left side for the non-standard cosmologies may be worth further pursuit although they both lie within the error range of \citet{Trowland_13} signal. We also include these correlations for 1,2 and 5 Mpc/h

\section{Discussion}\label{discussion}

\section{Conclusion}\label{conclusion}

\bibliographystyle{mnras}
\bibliography{biblio} 

% Don't change these lines
\bsp	% typesetting comment
\label{lastpage}
\end{document}

% End of mnras_template.tex
